\documentclass[a4paper, 11pt]{article}
\usepackage{mathtools}
\usepackage{amssymb}
\usepackage[russian]{babel}
\usepackage[utf8]{inputenc}
\usepackage{textcomp}
\begin{document}

\quad\quad Задача 1:\\
Доказать: $\frac{n^3}{6} - 7n^2 = \Omega(n^3)$\\
Воспользуемся определением:\\
$\exists\medspace c \in \mathbb{R},\medspace c > 0,\medspace N \in \mathbb{N}: \forall n \in \mathbb{N}: n > N,\medspace\frac{n^3}{6} - 7n^2 > cn^3$\\
Возьмем $c = \frac{1}{12}$, разделим выражение на $n^2$ и умножим на $12$.\\
$n > 84$\\
Т.е. при $n > 84$ это будет выполняться. Тогда возьмем $N = 84$.\\
Формула будет справедлива при таких $c, N \Rightarrow \frac{n^3}{6} - 7n^2 = \Omega(n^3)$ 
\\

\quad\quad Задача 2:\\
$
1, \medspace n^\frac{1}{\log{n}}, \medspace (\frac{3}{2})^2, \medspace \log{\log{n}}, \medspace \sqrt{\log{n}}, \medspace \log^2{n}, \medspace (\sqrt{2})^{\log{n}}, \medspace n, \medspace 2^{\log{n}}, \medspace \log{n!}, \medspace n\log{n}, \medspace n^2\\ \medspace 4^{\log{n}}, \medspace n^3, \medspace (\log{n})!, \medspace n^{\log{\log{n}}}, \medspace (\log{n})^{\log{n}}, \medspace n\cdot2^n, \medspace e^n, \medspace n!, \medspace (n + 1)!, \medspace 2^{2^n}, \medspace 2^{2^{n + 1}}
$
\\

\quad\quad Задача 3:\\
Очевидно что $\log{n!} = \mathcal{O}(n\log{n})$\\
$\log{n!} = \log{1} + \cdots + \log{n} \leqslant n\log{n}$\\
Докажем что $\log{n!} = \Omega(n\log{n})$\\
$\log{1} + \cdots + \log{\frac{n}{2}} + \log(\frac{n}{2} + 1) + \cdots +
\log{n} > \frac{n}{2}\log{\frac{n}{2}}\\
\frac{n}{2}\log{\frac{n}{2}} > cn\log{n}$\\
Пусть: $c = \frac{1}{4}$\\
Тогда сократим на $n$ и отбросим возрастающий $\log$\\$\frac{n}{2} > \sqrt{n}$\\
Это верно при $n > 4 \Rightarrow \log{n!} = \Omega(n\log{n}) \Rightarrow
\log{n!} = \Theta(n\log{n})$\\

\quad\quad Задача 4:\\
Построим дерево рекурсии и посчитаем сумму операций.\\
Поймем что сумма равна $n\log{\log{n}}$\\
Докажем по индукции\\
База индукции очевидна\\
$T(c) = \mathcal{O}(1)$\\
Предположение\\
$T(n) = cn\log{\log{n}} = \mathcal{O}(n\log{\log{n}})$\\
Переход\\
$T(n) = cn\log\log{\frac{n}{2}} + \frac{n}{\log{n}} < cn\log\log{n}$\\
$c\log{n}(\log\log{n} - \log\log{\frac{n}{2}}) > 1$\\
$c\log{n}(\log\frac{\log{n}}{\log{\frac{n}{2}}}) > 1$\\
$\log{(\frac{\log{n}}{\log{\frac{n}{2}}})^{c\log{n}}} > 1$\\
$(\frac{\log{n}}{\log{n} - 1})^{c\log{n}} > 2$\\
Сделаем замену: $k = \log{n} - 1$\\
Пусть: $ \space c = 1$\\
$f_1(k) = (1 + \frac{1}{k})^{k + 1} > 2$\\
$\lim_{k \to \infty}f_1(k) = e$\\
Причем $\forall k > 2, \space f_1(k) > e$ так как функция убывает (известный факт)\\
$(1 + \frac{1}{k})^{k + 1} > 2 \Rightarrow
T(n) = \mathcal{O}(n\log{\log{n}})$\\
Теперь докажем что $T(n) = \Omega(n\log{\log{n}})$\\
Аналогичные рассуждения индукции\\
Возьмем $c = \frac{1}{8}$\\
$f_2(k) = (1 + \frac{1}{k})^{\frac{k + 1}{8}} < 2$\\
$\lim_{k \to \infty}f_2(k) = \sqrt[8]{e}$\\
Причем $\forall k > 2, \space f_1(k) > \sqrt[8]{e} \Rightarrow 
\exists k_1 : f_2(k_1) < \sqrt[8]{e} < 2 \Rightarrow\\
\Rightarrow T(n) = \Omega(n\log{\log{n}}) \Rightarrow
T(n) = \Theta(n\log{\log{n}})$\\

\quad\quad Задача 5:\\
Рассмотрим первый алгоритм:\\
На каждом шаге оба аргуманта делятся на $2$ так как в качестве $k$ выбирается $\frac{n}{2}$. Значит $\max(n_1, n_2)$ каждый раз будет уменьшаться вдвое. Каждый вызов функции делает $\mathcal{O}(n)$ операций и рекурсивно вызывается 4 раза. Тогда можно записать формулу количества операций.\\
$T_1(n) = 4 \cdot T_1(\frac{n}{2}) + c \cdot n$\\
Константу можно вынести.\\
Применим Мастер теорему к $T_1(n)$\\Получим что $T_1(n) = cn^2 \Rightarrow T_1(n) = \Theta(n^2)$\\
Рассмотрим второй алгоритм:\\
Рассуждения аналогичны. Разница есть лишь в количестве вызовов рекурсии, их здесь $3$.\\
$T_2(n) = 3 \cdot T_2(\frac{n}{2}) + c \cdot n$\\
Выносим константу и используем Мастер теорему.\\
$T_2(n) = cn^{\log_{2}{3}} \Rightarrow T_2(n) = \Theta(n^{\log_{2}{3}})$

\end{document}

