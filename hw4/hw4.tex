\documentclass[a4paper, 10pt]{article}
\usepackage{mathtools}
\usepackage{verbatim}
\usepackage{amssymb}
\usepackage[russian]{babel}
\usepackage[utf8]{inputenc}
\usepackage{textcomp}
\usepackage{listings}

\begin{document}

\begin{enumerate}

    \item{
        Найдем $k: 2^k < p, 2^{k + 1} >= p$\\
        Его можно легко найти за $\mathcal{O}(\log_2{p})$ просто итеративно увеличивая $k$ и проверяя условие.\\ 
        Теперь у нас есть отрезок отсортированного массива длины $\mathcal{O}(p)$ на котором нужно найти элемент. Сделать это можно бинпоиском за $\mathcal{O}(log_2{p})$\\
    }

    \item{
        Это противная задача. Давайте научимся решать другую.\\
        А именно, найдем количество пар $(L,R)$, таких что на этом отрезке не более $k$ различных чисел.\\
        Пусть мы зафиксировали левую границу $L$, тогда пусть $R(L)$ "--- минимальная граница, такая что на $[L; R(L)]$ больше чем $k$ различных чисел.\\
        Получается, для фиксированной границы $L$, ответом будет $R(L) - L$\\
        Заметим что $\forall L < n \medspace R(L) \leq R(L + 1)$\\
        Давайте запомним $R(L)$ и будем считать $R(L + 1)$ используя предыдущее.\\
        Нужно ещё как-то поддерживать количество различных элементов на отрезке, а так же уметь увеличивать правую и левую границы.\\
        Заведём массив подсчета $cnt$. Будем в нем хранить количество элементов по значению.\\
        Тогда если мы добавляем $a_i$ и $cnt_{a_i} = 0 \implies $ у нас добавился новый уникальный элемент.
        Аналогично при удалении: если удаляем элемент, а он один в отрезке, то значит кол-во различных уменьшилось на $1$.\\
        Каждый раз мы двигаем правую границу отрезка, но суммарно более $n$ раз мы подвинуть не можем, следовательно амортизированное время работы $\mathcal{O}(n)$.\\
        Посчитали количество пар $(L, R)$ таких что количество различных чисел на этом отрезке не более $k$.\\
        Посчитаем тоже самое для $k - 1$.\\
        Вуаля, если вычтем из первого второе, то получим количество пар $(L, R)$ таких что на этом отрезке ровно $k$ различных чисел.\\
    }

    \item{
        Вот у нас есть два массива $a, b$ размерами $n, m$ и мы ищем $k$-ю порядковую статистику. Далее нумерация будет с $1$\\
        Пусть $i = \frac{n}{2}, j = k - \frac{n}{2}$\\
        Заметим что $i + j = k$\\
        Посмотрим на $a_i, b_j$\\
        Если $b_j < a_i < b_{j + 1}$, то ответ $a_i$, потому что он стоит ровно на месте $i + j = k$\\
        Если $a_i < b_j < a_{i + 1}$, то ответ $b_j$ (аналогично)\\
        Теперь, если оба условия не выполнились и $a_i > b_j$, то $a_i > b_{j + 1}$, так как иначе было бы верно второе условие. Отсюда видно что никакой из элементов до $b_j$ включительно (в массиве $b$ конечно) не может быть $k$-й порядковой статистикой, так как все они будут с индексами меньшими $i + j$. Можно заметить что и все элементы большие $a_i$ (в массиве $a$) тоже не будут $k$-й порядковой статистикой, потому что будут иметь индексы большие $i + j$ в массиве слияния.\\
        Тогда мы можем скипнуть $j$ элементов массива $b$ и искать $k - j$ порядковую статистику на массивах $a[1 : i - 1], b[j + 1 : m]$\\
        Симметричный случай, когда $b_j > a_i$, разбирается аналогично.\\
        Что делать, когда один из массивов длины $1$?\\
        Пусть $n = 1$, тогда если $b_{k - 1} < a_0 \Rightarrow$ ответ "--- $a_0$, иначе ответ "--- $b_{k - 1}$\\
        Если $m = 1$, то поменяем массивы местами и всё аналогично.\\
        Каждый уровень рекурсии $n$ уменьшается в $2$ раза, значит если передать первым массивом "--- меньший из двух, то алгоритм будет работать за $\mathcal{O}(\log_2(\min(n, m)))$
    }

\end{enumerate}

\end{document}