\documentclass[14pt,a4paper,report]{ncc}
\usepackage[a4paper, mag=1000, left=2.5cm, right=1cm, top=2cm, bottom=2cm, headsep=0.7cm, footskip=1cm]{geometry}
\usepackage{mathtools}
\usepackage{amsmath}
\usepackage{verbatim}
\usepackage{amssymb}
\usepackage[utf8]{inputenc}
\usepackage{textcomp}
\usepackage[english,russian]{babel}
\usepackage{indentfirst}
\usepackage[dvipsnames]{xcolor}
\usepackage[colorlinks]{hyperref}
\usepackage{listings} 
\usepackage{caption}

\begin{document}

\textbf{Задача 1}\\
Пусть это правда и в \underline{сортирующей} сети нет компаратора $(i, i+1)$.\\
Возьмём последовательность $0_1, \ldots, 0_{i-1}, 1_i, 0_{i+1}, 1_{i+2}, \ldots , 1_n$\\
Она не сортируется этой сетью, потому что никакой компаратор кроме $(i, i+1)$ не меняет последовательность, а нужного компаратора в сети нет.\\
Нашли противоречие $\Rightarrow$ в любой сортирующей сети есть компаратор $(i, i+1)$\\

\textbf{Задача 2}\\
Пусть мы сливаем нить номер $1$ с нитями $2, 3, \ldots, n$\\
Посмотрим какими могут быть компараторы на первом слое сети.\\
Очевидно что компаратор будет один и он должен соединять нить $1$ с какой-то нитью $k$\\
Докажем что нужно не менее $\log_2{n}$ компараторов что-бы добавить элемент:\\
\begin{itemize}
    \item {
        \textit{База индукции}:\\
        Для $n = 2$ нам нужен $1$ компаратор и $1 \geq \log_2{2}$
    }
    \item {
        \textit{Переход}:\\
        Пусть $\forall n' < n$ условие выполнено.\\
        Единственный доступный нам первый компаратор сравнивает первый элемент с элементом номер $k$\\
        Теперь есть 2 случая:
        \begin{itemize}
            \item {Они поменялись, тогда мы знаем что первый больше чем $k$-й. Тогда задача свелась к 2 задачам.\\Нужно вставить (теперь уже) первый элемент в отрезок массива $[2, 3, \ldots, k - 1]$ и (теперь уже) $k$-й элемент в отрезок $[k + 1, \ldots, n]$}
            \item {Они не поменялись, тогда мы знаем что первый не больше чем $k$-ый. Тогда, казалось бы, можно теперь вставлять первый в отрезок $[2, \ldots, k - 1]$, но нет.\\
            Мы не может отличить этот случай от второго и обязаны вставить $k$-й в $[k + 1, \ldots, n]$}
        \end{itemize}
        Помним что мы умеем втавлять элемент в множество размера $n - 1$ не менее чем за $\log_2{n}$\\
        Тогда глубина результата будет: $1 + \max(\log_2{k - 1}, \log_2{n - k} \geq \log_2{n}$\\
        Пусть $k - 1 > n - k$ (абсолютно не влияет на решение, второй случай симметрично)\\
        $2k - 1 > n$ (из "Пусть")\\
        $\log_2(2(k - 1)) \geq \log_2{n}$ (из неравенства с максимумом)\\
        $\log_2(2k - 2) \geq \log_2{n}$\\
        Тут уже очевидно что это прада.
    }
\end{itemize}

\textbf{Задача 3}\\
Пусть $k$ элементов из верхнего массива должны оказаться во втором.\\
Тогда ровно $k$ элементов из второго массива должны оказаться в первом.\\
Так как массивы отсорчены (пусть сортируем по возрастанию), то эти $k$ элементов в первом массиве находятся строго в конце, а во втором массиве "--- строго в начале.\\
Теперь очевидно соединим соответствующие индексы компараторами. Если компараторы будут соединять лишние индексы, то хуже не будет, так как если первый элемент находится в своей части, и второй тоже, то поменяться они не могут.

\end{document}