\documentclass[14pt,a4paper,report]{ncc}
\usepackage[a4paper, mag=1000, left=2.5cm, right=1cm, top=2cm, bottom=2cm, headsep=0.7cm, footskip=1cm]{geometry}
\usepackage{mathtools}
\usepackage{amsmath}
\usepackage{verbatim}
\usepackage{amssymb}
\usepackage[utf8]{inputenc}
\usepackage{textcomp}
\usepackage[english,russian]{babel}
\usepackage{indentfirst}
\usepackage[dvipsnames]{xcolor}
\usepackage[colorlinks]{hyperref}
\usepackage{listings} 
\usepackage{caption}
\everymath{\displaystyle}

\begin{document}

\textbf{Задача 1}\\
Пусть $dp_i$ "--- количество возрастающих последовательностей, заканчивающихся в $i$ элементе.\\
\textit{Важно}\\
Считать будем так, что бы $\forall i \neq j \\dp_i \cap dp_j = \varnothing$\\
Даже если $a_i = a_j$, это важно\\
То есть не считаем по несколько раз одни и теже подпоследовательности.\\
Забавно, но начальных значаний видимо нет (учтем при пересчёте)\\
\textit{Пересчет:}\\
Идем по возрастанию $i$ и считаем $dp_i$
Пусть текущий элемент (значение) $x$, пересчитываем $dp_i$.\\
Тогда два варианта:\\
\begin{enumerate}
    \item {
        $x$ встречается впервые.\\
        Тогда $dp_i = \sum_{j=0}^{i - 1}dp_j$
        Это правда, потому что все $dp_j$ попарно не пересекаются и не будут в итоге пересекаться с $dp_i$, в силу того что отличаются от него в последнем элементе.
    }
    \item {
        $x$ встречался ранее на позиции $k$.\\
        Тогда $dp_i = \sum_{j = k}^{i - 1}$\\
        Посмотрим почему так.\\
        Все последовательности, такие что $a[k + 1, i - 1]$ элементы не входят, уже посчитаны в $dp_k$, так как $a_k = x$. Потому нет смысла считать их дважды.\\
        Все последовательности включающие элементы $a[k, i - 1]$ и элемент $x$ будут уникальны и не будут пересекаться с $dp_i$, потому что отличаются в последнем элементе, либо имеют элемент из $a[k, i - 1]$.
        Сумму посчитаем префиксными суммами, которые в процессе будем заполнять.
    }
\end{enumerate}
Итого, если быстро умеем находить и обновлять последнее вхождение элемента, то и динамику такую насчитаем.\\
Последние вхождения храним в виде массива на $n$ элементов, каждый раз его обновляя.\\
Так как $dp_i$ попарно не пересекаются, то ответ на задачу "--- сумма всех $dp_i$\\
Все подпоследовательности учли, потому что перебрали все возможные конечные элементы.\\

\textbf{Задача 2} \textit{Ку-ка-ре-ку}\\
Тут надо кукарекнуть что динамика НОП при строке и развернутой строке эквивалентна динамике по поиску наибольшего подпалиндрома.\\
Как искать наибольший подпалиндром?\\
$dp_{i, j} = $ длина наибольшего подпалиндрома на $a[i, j]$\\

Тогда если $a_i = a_j$, то очевидно что $dp_{i, j} = dp_{i + 1, j - 1} + 2$\\
А если нет, то один из краёв в оптимальном ответе точно роли не играет и $dp_{i, j} = \max(dp_{i + 1, j}, dp_{i, j - 1})$\\
Начальные значения $\forall i, dp_{i, i} = 1$ (очевидно)\\
Пересчет в порядке увеличения величины $j - i$ (длины отрезка)\\
Как доказать узнаем на разборе.\\

\textbf{Задача 3} (Всем извесная \textit{баян-задача с иннополиса :)})\\
Насчитаем несколько штук.\\
$\operatorname{d1}, \operatorname{d2}, \operatorname{cnt1}, \operatorname{cnt2}$\\
$\operatorname{d1}$ "--- длина НВП на $\{a_0, a_1, \ldots, a_i\}$, если $a_i$ точно берём в ответ.\\
$\operatorname{d2}$ "--- тоже самое на $\{a_i, a_{i + 1}, \ldots, a_{n - 1}\}$\\
$\operatorname{cnt1}$ "--- количество НВП на $\{a_0, a_1, \ldots, a_i\}$, если $a_i$ точно входит в НВП.\\
$\operatorname{cnt2}$ "--- количество НВП на $\{a_i, a_{i + 1}, \ldots, a_{n - 1}\}$, если $a_i$ точно входит в НВП.\\
Фух, погнали писать\\
$\operatorname{d1}, \operatorname{d2}$ мы считать умеем (научились на парах).\\
$\operatorname{cnt1}$ сейчас считать научимся, а $\operatorname{cnt2}$ считается аналогично первому.\\

Вспомним как мы считали динамику для НВП за квадрат.\\
Нужно на префиксе найти элемент меньший нашего с максимальным значением $dp$.\\
Утверждается что $cnt_i = \sum_{j | a_j < a_i, dp_j \rightarrow max} cnt_j$\\
Как такое посчитать? Хороший вопрос.\\
Будем использовать дерево отрезков, которое мы ещё к сожалению не прошли (завуалированное под разделяй и властвуй, которое уже прошли).\\

Храним в доп. массиве $b$ по значению элемента ($a_i$), максимальную $dp$, заканчивающуюся на $a_i$ и сумму $cnt$ по таким $dp$\\
Теперь нам нужно найти на префиксе этого массива максимальный элемент по значению $dp$ и сумму $cnt$ по таким элементам.\\

Построим разделяй и властвуй и закэшируем ответы для каждого отрезка, который посетили. (Тут инструкция как ДО пишется).\\
Когда говорят обновить элемент "--- изменятся не более чем $\log_2{n}$ отрезков. Пересчитаем значания начиная от листа и заканчивая корнем.\\
Когда поступает запрос на отрезке найти минимум и сумму "--- жадно разобьем отрезки на подряд идущие степени двойки от старшей в начале отрезка (в нуле) к младшей в конце отрезка.\\
Легко заметить что все такие отрезки у нас есть сохранённые.\\

Как объединять ответ для двух отрезков?\\
Просто смотрим на максимумы\\
Равны? Тогда ответ "--- максимум $dp$ и сумма $cnt_l, cnt_r$\\
Не равны? Тогда ответ "--- один из максимумов по $dp$ и соответствующая $cnt$\\

Отлично, посчитали необходимые штуки. Пусть НВП всего массива имеет длину $L$ и количество таких подпоследовательностей всего $K$\\
Проходим каждый элемент.\\
Если $\operatorname{d1}_i + \operatorname{d2} - 1 < L$, то элемент не входит ни в одну НВП, потому что если его обязательно взять, то макс НВП не выйдет, как видно из неравенства.
Иначе если $\operatorname{cnt1}_i \cdot \operatorname{cnt2} < K$, то элемент лежит хотя бы в одной НВП, но не лежит во всех НВП сразу. Ну эта формула есть ровно количество НВП, при условии что элемент $i$ мы точно взяли (привет, комбинаторика)\\
Иначе третий вариант методом исключения.
В итоге посчитали за $\mathcal{O}(n\log{n})$ хорактеристику каждого элемента.
\end{document}