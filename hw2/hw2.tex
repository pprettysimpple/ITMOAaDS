\documentclass[a4paper, 10pt]{article}
\usepackage{mathtools}
\usepackage{verbatim}
\usepackage{amssymb}
\usepackage[russian]{babel}
\usepackage[utf8]{inputenc}
\usepackage{textcomp}
\usepackage{listings}
\usepackage{xcolor}
 
\definecolor{codegreen}{rgb}{0,0.6,0}
\definecolor{codegray}{rgb}{0.5,0.5,0.5}
\definecolor{codepurple}{rgb}{0.58,0,0.82}
\definecolor{backcolour}{rgb}{0.95,0.95,0.92}
 
\lstdefinestyle{mystyle}{
    backgroundcolor=\color{backcolour},   
    commentstyle=\color{codegreen},
    keywordstyle=\color{magenta},
    numberstyle=\tiny\color{codegray},
    stringstyle=\color{codepurple},
    basicstyle=\ttfamily\footnotesize,
    breakatwhitespace=false,         
    breaklines=true,                 
    captionpos=b,                    
    keepspaces=true,                 
    numbers=left,                    
    numbersep=5pt,                  
    showspaces=false,                
    showstringspaces=false,
    showtabs=false,                  
    tabsize=2
}
 
\lstset{style=mystyle}
\begin{document}

\begin{enumerate}
\item
\begin{enumerate}
	\itemИстинное время работы increment в худшем случае составит k операций\\
	Это значение достигается когда массив а заполнен единицами.\\

	\itemПосмотрим сколько раз мы инвертировали бит с номером i.\\
	Нулевой бит инвертируется каждую операцию, первый бит - каждую вторую.
	Нетрудно заметить что i бит меняет значение только после $2^{i}$ операций $\operatorname{increment}$.
	Итого получаем $n + \frac{n}{2} + \frac{n}{4} + ... \leq 2n$\\

	\itemРассмотрим массив в котором на $k-1$ месте стоит единица, а остальные значения - нули.\\
	Будем чередуя выполнять операции decrement и increment.\\
	Нетрудно заметить что каждая будет выполняться за k простых операций.\\
	Следовательно время работы в худшем случае --- $\mathcal{O}(nk)$.\\

	\item Заведём дополнительную переменную $\operatorname{right}$, изначально равную $-1$, --- самая правая единичка. После неё в массиве может храниться что угодно, но мы будем считать что там лежат нули.
	\begin{enumerate}
		\itemКогда цикл в $\operatorname{increment}$ дойдет до $\operatorname{right} + 1$ мы должны остановиться так как помним о том что все элементы после $\operatorname{right}$ равны нулю. Ещё нужно подвинуть $\operatorname{right}$ в случае когда $i~>~\operatorname{right}$, потому что самая правая единичка может подвинуться правее.
		\item$\operatorname{get(i)}$ нужно изменить так что-бы при $i > \operatorname{right}$ она возвращала $0$.
		\itemТеперь $\operatorname{setZero}$ будет просто ствить   $\operatorname{right}$ в значение $-1$
	\end{enumerate}
	\begin{lstlisting}[language=Python]
	increment():
		i = 0
		while i < k and i <= right and a[i] = 1:
			a[i] = 0
			i++
		right = max(right, i)
		if i < k:
			a[i] = 1

	get(i):
		if i > r:
			return 0; 
		return a[i]

	setZero():
		r = -1
	\end{lstlisting}
\end{enumerate}

\newpage

\item
\begin{enumerate}
	Придумаем функцию потенциала:\\
	$n$ "--- количество элементов в векторе\\
	$c$ "--- фактический размер вектора \\
	$\frac{c}{4} \leq n \leq c$\\
	$\Phi(n, c) = \begin{cases}
		c \leq 2n, 2n - c,\\
		2n < c, 2c - n.
	\end{cases}$\\
	$\operatorname{push} = \begin{cases}
		c \leq 2n, \medspace 1 + \Phi(n + 1, c) - \Phi(n, c) = \mathcal{O}(1),\\
		2n < c, \medspace 1 + \Phi(n + 1, c) - \Phi(n, c) = \mathcal{O}(1),\\
		n = c, \medspace n + \Phi(n + 1, 2c) - \Phi(n, c) =\\
		= n + 2n + 2 - 2c - 2n + c = 3 + n - c = \mathcal{O}(1).\\
	\end{cases}$\\
	$\operatorname{pop} = \begin{cases}
		c \leq 2n, \medspace 1 + \Phi(n + 1, c) - \Phi(n, c) = \mathcal{O}(1),\\
		2n < c, \medspace 1 + \Phi(n + 1, c) - \Phi(n, c) = \mathcal{O}(1),\\
		n = \frac{c}{4}, \medspace c + \Phi(n - 1, \frac{c}{2}) - \Phi(n, c) =\\
		= c + c - n + 1 - 2c + n = 1 = \mathcal{O}(1).\\
	\end{cases} \Rightarrow \\ \Rightarrow$ все операции выполняются в среднем за $\mathcal{O}(1)$ 
\end{enumerate}

\item
\begin{enumerate}
	Пусть у нас есть неинициализированный массив $a$.
	Заведём дополнительно массив $b$ на $n$ элементов и стек на массиве $s$.
	$s$ будет соответствовать элементам, которым мы уже что-нибудь присвоили. В ячейке стека будем хранить позицию в массиве, куда мы присваивали.\\
	В $a$ будем хранить последнее значение, либо мусор. Во $b$ храним позицию в стеке нашего элемента, либо опять мусор.\\
	Как это работает?\\
	При запросе значения в ячейке $i$, будем смотреть на значение в $b_i$. Если $b_i \geq \operatorname{size}(s)$, то очевидно в эту ячейку мы ничего не присваивали и нужно вернуть из $\operatorname{get}(i)$ нуль.\\
	Посмотрим на случай, когда $b_i < \operatorname{size}(s)$.\\
	Мы точно знаем что внутри стека нет мусора и каждый элемент соответствует инициализированному значению.\\
	Это значит что $s_{b_i} = i \iff i$ "--- была проинициализирована ранее.\\
	Осталось поддерживать эту структуру при добавлении элемента, это просто.\\
	В $\operatorname{set}(i, x)$ есть два случая.\\
	$a_i$ "--- проинициализирована, тогда нужно просто записать в $a$ новое значение $x$.\\
	$a_i$ "--- не проинициализирована, тогда нужно записать в $b_i$ значение $\operatorname{size}(s)$.
	Затем добавить в $s$ вершину со значением $i$ и присвоить элементу $a_i$ значение $x$.\\
	\newpage
	Вот код для понимания.\\
	\begin{lstlisting}[language=Python]
	a, b, s = [], [], []
	sSize = 0
	init(n):
		a, b, s = [n], [n], [n]

	isInit(i):
		if b[i] < sSize and s[b[i]] == i:
			return True
		else:
			return False

	get(i):
		if isInit(i):
			return a[i]
		else:
			return 0

	set(i, x):
		if isInit(i):
			a[i] = x
		else
			a[i] = x
			s[sSize] = i
			b[i] = sSize
			sSize++;
	\end{lstlisting}
\end{enumerate}

\item
\begin{enumerate}
	Будем внутри нашей памяти хранить односвязный список не занятых блоков. Заведём указатель $\operatorname{head}$ на последний элемент списка. В вершине списка будем хранить одно число "--- указатель на предыдущий элемент, или $-1$, если элемент является последним.\\
	Пусть для нас память представляет собой массив $a$ длины $n$.
	Проинициализируем её так:\\$\forall i \in \{0, 1, \medspace \ldots, n - 1\}, \medspace a_i = i - 1$\\
	$\operatorname{head} = n - 1$\\
	Теперь очевидно. Когда нужно вернуть номер свободной ячейки, мы возвращаем номер $\operatorname{head}$ и удаляем последний элемент списка.\\
	Когда нам говорят сделать $\operatorname{free}$ некоторой ячейки, мы добавляем эту вершинку в наш список неиспользованных.\\
	\begin{lstlisting}[language=Python]
	head = -1
	mem = []
	init(n):
		mem = [n]
		for i in range(0, n):
			mem[i] = i - 1
		head = n - 1

	malloc():
		tmp = head
		head = mem[head]
		return tmp

	free(p):
		mem[p] = head
		head = p\end{lstlisting}
\end{enumerate}

\item
\begin{enumerate}
	Найдем и докажем время работы $\operatorname{push}$\\
	Пусть $\{a_1, \medspace a_2, \medspace \ldots, a_k\} = A$\\
	Введём функцию потенциала:\\
	$\Phi(A) = \sum_{i=0}^{k}(k - i)2^i$
	Тогда рассмотрим что происходит при добавлении одного элемента в структуру.\\
	Пусть в структуре первые $k'$ массивов не пусты.\\
	Тогда элемент, который мы вставляем в структуру, $k'$ раз\\
	поучаствует в $\operatorname{merge} \Rightarrow $ на это мы потратим $k'$ операций.\\
	$\operatorname{push}$ работает за:\\
	$(1)k' + \Phi(A') - \Phi(A) = k' + (k - k' - 1)2^{k' + 1} - \sum_{i=0}^{k'}(k - i)2^i$\\
	Рассмотрим сумму поближе:\\
	$(2)\sum_{i=0}^{k'}(k - i)2^i = k2^{k' + 1} - \sum_{i=0}^{k'}i2^i$\\
	Ещё ближе, теперь уже на сумму $\sum_{i=0}^{k}i2^i$.\\
	Докажем по индукции что она равна $2(k2^k - 2^k + 1)$\\
	$2(k2^k - 2^k + 1) + (k + 1)2^{k + 1} = k2^{k + 1} - 2^{k + 1} + 2 + 2^{k + 1} + k2^{k + 1} = \\
	= k2^{k + 2} + 2^{k + 2} - 2^{k + 1} + 2 = 2((k + 1)2^{k + 1} - 2^{k + 1} + 1)$\\
	Доказали, круто, вернёмся слегка назад к (2).\\
	$k2^{k' + 1} - \sum_{i=0}^{k'}i2^i = k2^{k' + 1} - k'2^{k' + 1} - 2^{k' + 1} + 2 = \\
	= 2^{k' + 1}(k - k' - 1) + 2$\\
	Окэй, посчитали (2), тогда можем почитать и (1).\\
	$k' + 2^{k' + 1}(k - k' - 1) - 2^{k' + 1}(k - k' - 1) + 2 = \mathcal{O}(k)$\\
	Значит $\operatorname{push}$ работает в среднем за $\mathcal{O}(k)$.\\
	\quad С $\operatorname{contains}$ всё проще, потому что структуру данных эта функция не изменяет (разность потенциалов равна нулю). Значит просто считаем худший случай.\\
	На $i$-й итерации запускается бинарный поиск на массиве\\размера $2^i$.\\
	$\operatorname{contains}$ работает за:\\
	$\log_2{2^0} + \log_2{2^1} + \log_2{2^2} + \cdots + \log_2{2^k} = \frac{k(k + 1)}{2} = \mathcal{O}(k^2)$\\
\end{enumerate}

\end{enumerate}

\end{document}