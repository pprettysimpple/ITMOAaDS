\documentclass[14pt,a4paper,report]{ncc}
\usepackage[a4paper, mag=1000, left=2.5cm, right=1cm, top=2cm, bottom=2cm, headsep=0.7cm, footskip=1cm]{geometry}
\usepackage{mathtools}
\usepackage{amsmath}
\usepackage{verbatim}
\usepackage{amssymb}
\usepackage[utf8]{inputenc}
\usepackage{textcomp}
\usepackage[english,russian]{babel}
\usepackage{indentfirst}
\usepackage[dvipsnames]{xcolor}
\usepackage[colorlinks]{hyperref}
\usepackage{listings} 
\usepackage{caption}

\begin{document}

\textbf{Задача 1}\\
Нам дали кучу $A$\\
Переопределим операцию $extract\_min$ для $A$ (про остальные операции забудем, потому что не нужны)\\
При извлечении максимального элемента, если куча одна, то разделим её на две кучи и вернём значение в старом корне.\\
Пусть у нас есть $i$ куч, оставшиеся от $A$ после $i-1$ операций $extract\_min$.\\
Что-бы извлечь минимум, нужно найти кучу с минимальным элементом и проделать операцию как в случае с одной кучей.\\
Нужно искать кучу с минимальным корнем за "быстро". Заведём ещё одну кучу, в которую будем добавлять новые корни и извлекать минимальный корень.\\
Что-бы найти $k$-й в исходной куче, нужно сделать $k$ новых $extract\_min$.\\
Итого на каждом $extract\_min$ у нас во вспомогательной куче лежит не более $k$ элементов $\Rightarrow$ извлекаем минимум и добавляем вершины за $\mathcal{O}(\log_2{k})$\\
Так как у нас $k$ операций $extract\_min$, то алгоритм работает за $\mathcal{O}(k\log_2{k})$

\textbf{Задача 2}\\
Будем хранить в вершине новое поле "--- сколько нужно добавить к этому поддереву.\\
В нужный момент будем проталкивать эту информацию детям.\\
Когда нужно добавить в поддерево $v$ число $d$ "--- прибавляем $d$ в поле $\operatorname{add}$ вершинки $x$\\
Пусть у нас для кучи будет храниться сколько нужно прибавить\\

\textit{Merge}\\
Нужно пропушить значение в обоих корнях и только потом делать \operatorname{Merge}.







\textbf{Задача 3}\\
Создадим структуру из двух куч одинакового размера (при нечетном суммарном количестве "--- в левой больше элементов).\\ Правая "--- минимальная, левая "--- максимальная.\\
Инвариант: в правой куче все элементы больше чем в первой.\\
Из инварианта и структуры следует что в корне левой кучи всегда будет находиться искомая медиана.\\

\textit{Добавляем}
\begin{itemize}
    \item {
        При добавлении элемента $x$ будем смотреть как он соотносится с корнем левой кучи $l$.\\
        $x > l, \Rightarrow$ добавим $x$ в правую кучу.\\
        $x \leq l, \Rightarrow$ добавим $x$ в левую кучу.\\

    }
    \item {
        Теперь нарушился инвариант на соотношение количеств элементов в кучах.
        Ну давайте просто из той, в которой много извлечем $\operatorname{max}(\operatorname{min})$ элемент из той кучи, в которой слишком много элементов и перекиенм в ту, в которой их не хватает.\\
        Так как баланс изменился не более чем н $1$ (мы добавили всего один элемент), то и перекинем таким образом мы не более одного элемента.
    }
\end{itemize}

\textit{Удаляем медиану}\\
Сделаем $extract\_min$ из левой кучи и как при добавлении перекинем нужный элемент из большей кучи в меньшую.\\

\textit{Возвращаем медиану}\\
Ну теперь совсем изи. По следствию у нас в корне левой кучи ровно медиана.

\end{document}